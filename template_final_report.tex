% --------------------------------------------------------------------------
%
% LaTeX Document for Computer Science Academic Report
% Template designed by Christopher Dedman Rollet in Spring 2025
% This document is a template for writing an academic report in LaTeX.
% You are free to use and modify this template for your own projects.
%
% Author: Christopher Dedman Rollet
% Date: Spring 2025
% Version: 1.0
% License: MIT License (see LICENSE file for details)
%
% Dependencies:
% - setspace package for double spacing
% - pdfpages package for including PDF pages
% - listings package for code listings
% - xcolor package for color support
%
% Usage Instructions:
% 1. Ensure you have the required packages installed.
% 2. Replace the placeholder text with your content.
% 3. Compile the document using pdflatex.
%
% Contact:
% - Website: http://chrisdedman.vercel.app 
%
% --------------------------------------------------------------------------

\documentclass[12pt]{report}
\usepackage{graphicx} % For including images
\usepackage{geometry} % For page layout
\usepackage{fancyhdr} % For custom headers and footers
\usepackage{setspace} % For line spacing
\usepackage{booktabs} % For better tables
\usepackage{amsmath}  % For mathematical formatting
\usepackage{tabularx} % For tables with automatic width
\usepackage{listings} % For code highlighting
\usepackage{xcolor}   % For code highlighting
\usepackage{pdfpages} % For including PDF pages
\usepackage{caption}  % For customizing captions
\usepackage{hyperref} % For hyperlinks (keep this package at the end)

% The next few lines are for setting up the layout and formatting
\geometry{letterpaper, margin=1in}      % Page layout
\setlength{\parskip}{1em}               % Space between paragraphs
\setlength{\parindent}{2em}             % Set the indentation length
\pagestyle{fancy}                       % Custom header and footer
\fancyhf{}                              % Clear header and footer
\rhead{\thepage}                        % Right header with page number
\cfoot{}                                % Center footer empty
\renewcommand{\headrulewidth}{0.1pt}    % Header rule width
\setlength{\headheight}{15pt}           % Header height
\setlength{\headsep}{10pt}              % Space between header and text
\setlength{\textheight}{9in}            % Text height
\setlength{\textwidth}{6.5in}           % Text width
\setlength{\topmargin}{-0.5in}          % Top margin
\setlength{\oddsidemargin}{0in}         % Left margin
\setlength{\evensidemargin}{0in}        % Right margin

% Code style with listings
\definecolor{codegray}{gray}{0.95}
\definecolor{darkergreen}{RGB}{0,100,0}
\lstset{
    basicstyle=\ttfamily\scriptsize,
    keywordstyle=\color{blue}\bfseries,
    stringstyle=\color{red},
    commentstyle=\color{darkergreen}\itshape,
    numbers=left,
    numberstyle=\tiny\color{gray},
    stepnumber=1,
    numbersep=5pt,
    backgroundcolor=\color{codegray},
    showspaces=false,
    showstringspaces=false,
    showtabs=false,
    frame=lines,
    framerule=0.5pt,
    tabsize=2,
    captionpos=b,
    breaklines=true,
    keepspaces=true,
    breakatwhitespace=false,
    escapeinside={\%*}{*)}
}

% Correct bad hyphenation here
\hyphenation{op-tical net-works semi-conduc-tor}

% Custom section, subsection, and subsubsection numbering
\renewcommand{\thesection}{\Roman{section}.}
\renewcommand{\thesubsection}{\Alph{subsection}.}
\renewcommand{\thesubsubsection}{\arabic{section}.\arabic{subsection}.\alph{subsubsection}}

% Custom commands for formatting
\newcommand{\code}[1]{\texttt{\textcolor{blue}{#1}}}

% Center section titles and remove numbering
\makeatletter
\renewcommand{\section}{\@startsection{section}{1}{0pt}{3.5ex plus 1ex minus .2ex}{2.3ex plus .2ex}{\centering\normalfont\large\bfseries}}
\makeatother

% Format subsection titles with run-in style
\makeatletter
\renewcommand{\subsection}{\@startsection
  {subsection}{2}{0pt}        % name, level, indent
  {1.5ex plus .2ex}           % space *before* subsection
  {1ex plus .2ex}             % space *after* subsection
  {\normalfont\scshape}       % style
}
\makeatother

% Format subsubsection titles with run-in style
\makeatletter
\renewcommand{\subsubsection}{\@startsection
  {subsubsection}{3}{1em}        % name, level, indent
  {1ex plus .2ex}                % space *before* subsubsection
  {0.5ex plus .2ex}              % space *after* subsubsection
  {\normalfont\itshape}          % style
}
\makeatother

% Customizing captions
\captionsetup{font=small, labelfont=bf}

\begin{document}

% ------------------------------------------------
% *** Cover Page ***
% ------------------------------------------------
% Uncomment the following line to include a cover page
% \includepdf[pages=1, pagecommand={\thispagestyle{empty}}]{cover_page.pdf}

% ------------------------------------------------
% *** Title Page ***
% ------------------------------------------------
% This section is for the title page
% You should add your project's title and your first and last name and your 
% semester and year. 
\doublespacing
\begin{titlepage}
    \begin{center}
        \vspace*{2in}
        {\Large \textbf{Project\_Title}}\\[0.2cm]
        {\Large Final Report}\\[0.2cm]
        \textit{A Project}\\[0.2cm]
        \textit{Presented}\\[0.2cm]
        \textit{To the Faculty of}\\[0.2cm]
        \textbf{California State University, Dominguez Hills}\\[0.2cm]
        \textit{In Partial Fulfillment of the Requirements}\\[0.2cm]
        \textit{For the Degree Bachelor of Science in Computer Science}\\[2cm]
        \textbf{By}\\[0.2cm]
        Your\_First\_Name Your\_Last\_Name\\[0.2cm]
        Semester Year
    \end{center}
\end{titlepage}

\setcounter{page}{2} % Set the page number to 2 after the title page

% ------------------------------------------------
% *** Approval Page ***
% ------------------------------------------------
% This section is for the approval page
% You should add your project's title and your first and last name in
% capital letters.
\newpage
\thispagestyle{empty}
\text{} \\
\textbf{PROJECT:} PROJECT\_TITLE\_IN\_CAPITAL\_LETTER\\[0.2cm]
\textbf{AUTHOR:} YOUR\_FIRST\_NAME YOUR\_LAST\_NAME

\vspace{6in}
\hfill \textbf{APPROVED BY:}\\ \\
\vspace{2cm}
\hfill \noindent\rule{6cm}{0.4pt}

% ------------------------------------------------
% *** Acknowledgements ***
% ------------------------------------------------
% This section is for acknowledgements or dedications
\newpage
\section*{Acknowledgements}
\addcontentsline{toc}{section}{Acknowledgements}
I would like to thank ...

% ------------------------------------------------
% *** Table of Contents and List of Figures ***
% ------------------------------------------------
% This section is for the table of contents and list of figures 
% It should automatically be updated as you add sections and figures
% You do not need to manually add sections to the table of contents
\singlespacing
\newpage
\tableofcontents
\newpage
\listoffigures
\newpage
\listoftables

% ------------------------------------------------
% *** Abstract ***
% ------------------------------------------------
% This section is for the abstract of your report
\doublespacing
\newpage
\section*{Abstract}
\addcontentsline{toc}{section}{Abstract}
This is the abstract of your project report. It should summarize the main points of your project, including the problem you are addressing, your approach, and your results.

% -----------------------------------------------------
% *** Main Content ***
% -----------------------------------------------------
% ------------------------------------------------
% *** Section 1: Introduction ***
% ------------------------------------------------
% This section can be used for introduction or background information
\newpage
\section{Introduction}
\subsection{Background}
Add background information here...

\subsection{Project Description}
Your project description goes here...

\subsection{Target Audience}
Add target audience information here...

\subsection{Motivation}
Add motivation information here...

% ------------------------------------------------
% *** Section 2: Related Work ***
% ------------------------------------------------
% This section can be used for related work or literature review
\newpage
\section{Related Work}
\subsection{Related Work 1}
Add related work 1 information here...

\subsection{Related Work 2}
Add related work 2 information here...

\subsection{Related Work n}
Add related work n information here...

% ------------------------------------------------
% *** Section 3: Other Section Title ***
% ------------------------------------------------
% This section can be used for any other information
% You can add any other section title and content here
% You can as well duplicated this section for more sections
\newpage
\section{Other Section Title}
\subsection{Other Subsection Title}
Add other subsection information here if needed...

% ------------------------------------------------
% *** Section 4: Methodology ***
% ------------------------------------------------
% This section can be used for any methodology or analysis
\newpage
\section{Methodology}
Methodology information goes here...

% ------------------------------------------------
% *** Section 5: Discussion ***
% ------------------------------------------------
% This section can be used for any additional discussion or analysis
\newpage
\section{Discussion}
Discussion information goes here...

% ------------------------------------------------
% *** Section 6: Conclusion ***
% ------------------------------------------------
% This section can be used for conclusion or summary
\newpage
\section{Conclusion}
Conclusion information goes here...

% ------------------------------------------------
% *** Section 7: Future Work ***
% ------------------------------------------------
% This section can be used for future work or next steps
\newpage
\section{Future Work}
Future work information goes here...

% ------------------------------------------------
% *** Section 8: References ***
% ------------------------------------------------
% This section can be used for references
\newpage
\addcontentsline{toc}{section}{References}
\begin{thebibliography}{99}
    \bibitem{key1} Author Name. \textit{Title of the Work}. Journal/Conference, Year.
    \bibitem{key2} ...
\end{thebibliography}

% ------------------------------------------------
% *** Section 9: Code ***
% ------------------------------------------------
% This section can be used for code snippets or examples
\newpage
\section{Code}
\begin{lstlisting}[language=Python, caption={Python Code Snippet}]
def example_function():
    # This is a comment
    print("This is an example function.")
    return True
\end{lstlisting}

\begin{lstlisting}[language=Go, caption={Golang Code Snippet}]
package main
import "fmt"
func main() {
    fmt.Println("This is an example Golang function.")
}
\end{lstlisting}

\begin{lstlisting}[language=C++, caption={C++ Code Snippet}]
#include <iostream>
using namespace std;
int main()
{
    // This is a comment
    cout << "This is an example C++ function." << endl;
    return 0;
}
\end{lstlisting}

\end{document}
