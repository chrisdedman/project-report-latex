% --------------------------------------------------------------------------
%
% LaTeX Document for Computer Science Academic Report
% Template designed by Christopher Dedman Rollet in Spring 2025
% This document is a template for writing an academic report in LaTeX.
% You are free to use and modify this template for your own projects.
%
% Author: Christopher Dedman Rollet
% Date: Spring 2025
% Version: 1.0
% License: MIT License (see LICENSE file for details)
%
% Dependencies:
% - setspace package for double spacing
% - pdfpages package for including PDF pages
% - listings package for code listings
% - xcolor package for color support
%
% Usage Instructions:
% 1. Ensure you have the required packages installed.
% 2. Replace the placeholder text with your content.
% 3. Compile the document using pdflatex.
%
% Contact:
% - Website: http://chrisdedman.vercel.app 
%
% --------------------------------------------------------------------------
\documentclass[12pt]{report}
\usepackage{graphicx} % For including images
\usepackage{geometry} % For page layout
\usepackage{fancyhdr} % For custom headers and footers
\usepackage{setspace} % For line spacing
\usepackage{booktabs} % For better tables
\usepackage{amsmath}  % For mathematical formatting
\usepackage{tabularx} % For tables with automatic width
\usepackage{listings} % For code highlighting
\usepackage{xcolor}   % For code highlighting
\usepackage{pdfpages} % For including PDF pages
\usepackage{float}    % For controlling figure placement
\usepackage{hyperref} % For hyperlinks
\usepackage{caption}  % For customizing captions

% Layout: 1-inch margins all around
\geometry{letterpaper, margin=1in}

% Remove conflicting layout settings
\setlength{\parskip}{1em}
\setlength{\parindent}{2em}

% Define preliminary page style (center bottom)
\fancypagestyle{prelim}{
  \fancyhf{}
  \cfoot{\thepage}
  \renewcommand{\headrulewidth}{0.1pt}
  \renewcommand{\footrulewidth}{0pt}
  \setlength{\footskip}{0.5in} % page number 0.5" from bottom
}

% Define main body page style (top right)
\fancypagestyle{main}{
  \fancyhf{}
  \rhead{\thepage}
  \renewcommand{\headrulewidth}{0pt}
  \setlength{\headsep}{0.5in} % page number 0.5" from top
}

% Code style
\definecolor{codegray}{gray}{0.95}
\definecolor{darkergreen}{RGB}{0,100,0}
\lstset{
    basicstyle=\ttfamily\scriptsize,
    keywordstyle=\color{blue},
    stringstyle=\color{red},
    commentstyle=\color{darkergreen},
    numbers=left,
    numberstyle=\tiny\color{gray},
    stepnumber=1,
    numbersep=5pt,
    backgroundcolor=\color{codegray},
    showspaces=false,
    showstringspaces=false,
    showtabs=false,
    frame=lines,
    framerule=0.5pt,
    tabsize=2,
    captionpos=b,
    breaklines=true,
    keepspaces=true,
    breakatwhitespace=false,
    escapeinside={\%*}{*)}
}

\hyphenation{op-tical net-works semi-conduc-tor}

\renewcommand{\thesection}{\Roman{section}.}
\renewcommand{\thesubsection}{\Alph{subsection}.}
\renewcommand{\thesubsubsection}{\arabic{section}.\arabic{subsection}.\alph{subsubsection}}

\newcommand{\code}[1]{\texttt{\textcolor{blue}{#1}}}

\makeatletter
\renewcommand{\section}{\@startsection{section}{1}{0pt}{3.5ex plus 1ex minus .2ex}{2.3ex plus .2ex}{\centering\normalfont\large\bfseries}}
\makeatother

\makeatletter
\renewcommand{\subsection}{\@startsection
  {subsection}{2}{0pt}        % name, level, indent
  {1.5ex plus .2ex}           % space *before* subsection
  {1ex plus .2ex}             % space *after* subsection
  {\normalfont\scshape}       % style
}
\makeatother

\makeatletter
\renewcommand{\subsubsection}{\@startsection
  {subsubsection}{3}{1em}     % name, level, indent
  {1ex plus .2ex}             % space *before* subsection
  {0.5ex plus .2ex}           % space *after* subsection
  {\normalfont\itshape}       % style
}
\makeatother

% Customizing captions
\captionsetup{font=small, labelfont=bf}

\begin{document}

% ------------------------------------------------
% *** Cover Page ***
% ------------------------------------------------
% Comment the following line to remove the cover page
\includepdf[pages=1, pagecommand={\thispagestyle{empty}}]{figures/cover_page.pdf}

% ------------------------------------------------
% *** Title Page ***
% ------------------------------------------------
% This section is for the title page
% TO-DO: You should add your project's title and your first and last name and your 
% semester and year. 
\doublespacing
\phantomsection
\begin{titlepage}
    \begin{center}
        \vspace*{2in}
        {\Large \textbf{Project\_Title}}\\[0.2cm]
        {\Large Final Report}\\[0.2cm]
        \textit{A Project}\\[0.2cm]
        \textit{Presented}\\[0.2cm]
        \textit{To the Faculty of}\\[0.2cm]
        \textbf{California State University, Dominguez Hills}\\[0.2cm]
        \textit{In Partial Fulfillment of the Requirements}\\[0.2cm]
        \textit{For the Degree Bachelor of Science in Computer Science}\\[2cm]
        \textbf{By}\\[0.2cm]
        Your\_First\_Name Your\_Last\_Name\\[0.2cm]
        Semester Year
    \end{center}
\end{titlepage}

\setcounter{page}{2} % Set the page number to 2 after the title page

% ------------------------------------------------
% *** Approval Page ***
% ------------------------------------------------
% This section is for the approval page
% TO-DO: You should add your project's title and your first and last name in
% capital letters.
\newpage
\phantomsection
\addcontentsline{toc}{section}{Approval Page}
\thispagestyle{empty}
\text{} \\
\textbf{PROJECT:} PROJECT\_TITLE\_IN\_CAPITAL\_LETTER\\[0.2cm]
\textbf{AUTHOR:} YOUR\_FIRST\_NAME YOUR\_LAST\_NAME

\vspace{6in}
\hfill \textbf{APPROVED BY:}\\ \\
\vspace{2cm}
\hfill \noindent\rule{6cm}{0.4pt}

% ------------------------------------------------
% *** Acknowledgements ***
% ------------------------------------------------
% This section is for acknowledgements or dedications
\newpage
\section*{Acknowledgements}
\addcontentsline{toc}{section}{Acknowledgements}
I would like to thank ...

% ------------------------------------------------
% *** Table of Contents and List of Figures ***
% ------------------------------------------------
% This section is for the table of contents and list of figures 
% It should automatically be updated as you add sections and figures
% You do not need to manually add sections to the table of contents
\singlespacing
\newpage
\phantomsection
\addcontentsline{toc}{section}{Table of Contents}
\tableofcontents
\newpage
\phantomsection
\addcontentsline{toc}{section}{List of Figures}
\listoffigures
\newpage
\phantomsection
\addcontentsline{toc}{section}{List of Tables}
\listoftables

% ------------------------------------------------
% *** Abstract ***
% ------------------------------------------------
% This section is for the abstract of your report
\doublespacing
\newpage
\section*{Abstract}
\addcontentsline{toc}{section}{Abstract}
This is the abstract of your project report. It should summarize the main points 
of your project, including the problem you are addressing, your approach, and your results.

% -----------------------------------------------------
% *** Main Content ***
% -----------------------------------------------------
% ------------------------------------------------
% *** Section 1: Introduction ***
% ------------------------------------------------
% This section can be used for introduction or background information
\newpage
\section{Introduction}
\subsection{Background}
Add background information here...

\subsection{Project Description}
Your project description goes here...

\subsection{Target Audience}
Add target audience information here...

\subsection{Motivation}
Add motivation information here...

% ------------------------------------------------
% *** Section 2: Prior Related Work ***
% ------------------------------------------------
% This section can be used for prior related work or literature review
\newpage
\section{Prior Related Work}
\subsection{Prior Related Work 1}
Add related work 1 information here...

% Here is an example of using an image from the /figure folder.
\begin{figure}[H]
    \centering
    \includegraphics[width=0.5\textwidth]{figures/cs_csudh.png}
    \caption{Computer Science Departement at CSUDH (figure example)}
    \label{fig:CS_CSUDH_LOGO}
\end{figure}

\subsection{Prior Related Work 2}
Add related work 2 information here...

\subsection{Prior Related Work n}
Add related work n information here...

% ------------------------------------------------
% *** Section 3: Resources ***
% ------------------------------------------------
% This section can be used for resources or additional information
% You can add any other section title and content here
% You can as well duplicated this section for more sections
\newpage
\section{Resources}
\subsection{Additional Resources}
Add additional resources information here if needed...

% ------------------------------------------------
% *** Section 4: Use-Case Diagram ***
% ------------------------------------------------
% This section can be used for use-case diagrams or analysis
\newpage
\section{Use-Case Diagram}
Use-case diagram information goes here...

% Uncomment the following lines to include a use-case diagram
% This is an example of how to include a picture into your report
% \begin{figure}[H]
%     \centering
%     \includegraphics[width=0.5\textwidth]{figures/placeholder.png}
%     \caption{use-case diagram}
%     \label{fig:use_case_diagram}
% \end{figure}

\subsection{Actors}
Add actors information here...

\subsection{Use-Cases}
Add use-cases information here...

% ------------------------------------------------
% *** Section 5: Sequence Diagram ***
% ------------------------------------------------
% This section can be used for sequence diagrams or analysis
\newpage
\section{Sequence Diagram}
Sequence diagram information goes here...

% ------------------------------------------------
% *** Section 6: Class Diagram ***
% ------------------------------------------------
% This section can be used for class diagrams or analysis
\newpage
\section{Class Diagram}
Class diagram information goes here...

% ------------------------------------------------
% *** Section 7: Entity-Relationship Diagram ***
% ------------------------------------------------
% This section can be used for entity-relationship diagrams or analysis
\newpage
\section{Entity-Relationship Diagram}
Entity-relationship diagram information goes here...

% ------------------------------------------------
% *** Section 8: Frontend Developement ***
% ------------------------------------------------
% This section can be used for frontend development information
% You can add any other section title and content here
\newpage
\section{Frontend Development}
\subsection{Tools}
Add tools information here...

\subsection{Pages Navigation}
Add pages navigation information here...

\subsection{Frontend Development 1}
Add frontend development 1 information here...

\subsection{Frontend Development n}
Add frontend development n information here...

% ------------------------------------------------
% *** Section 9: Backend Development ***
% ------------------------------------------------
% This section can be used for backend development information
% You can add any other section title and content here
\newpage
\section{Backend Development}
\subsection{Tools}
Add tools information here...

\subsection{Backend Development 1}
Add backend development 1 information here...

\subsection{Backend Development n}
Add backend development n information here...

% ------------------------------------------------
% *** Section 10: Contributions ***
% ------------------------------------------------
% This section can be used for contributions or acknowledgements
% You can add any other section title and content here
\newpage
\section{Contributions}
Add contributions information here...

% ------------------------------------------------
% *** Section 11: Conclusion & Future Work ***
% ------------------------------------------------
% This section can be used for conclusion and future work
% You can add any other section title and content here
\newpage
\section{Conclusion \& Future Work}
Add conclusion and future work information here...

% ------------------------------------------------
% *** Section 12: Other Section ***
% ------------------------------------------------
% This section can be used for other information
% You can add any other section title and content here
\newpage
\section{Other Section}
Add other section information here...

% ------------------------------------------------
% *** Section 13: Code ***
% ------------------------------------------------
% This section can be used for code snippets or examples
\newpage
\section{Code}
\begin{lstlisting}[language=Python, caption={Python Code Snippet}]
def example_function():
    # This is a comment
    print("This is an example function.")
    return True
\end{lstlisting}

\begin{lstlisting}[language=Go, caption={Golang Code Snippet}]
package main
import "fmt"
func main() {
    fmt.Println("This is an example Golang function.")
}
\end{lstlisting}

\begin{lstlisting}[language=C++, caption={C++ Code Snippet}]
#include <iostream>
using namespace std;
int main()
{
    // This is a comment
    cout << "This is an example C++ function." << endl;
    return 0;
}
\end{lstlisting}

% ------------------------------------------------
% *** Section 14: References ***
% ------------------------------------------------
% This section can be used for references
\newpage
\phantomsection
\addcontentsline{toc}{section}{References}
\renewcommand{\bibname}{References}
\begin{thebibliography}{99}
    \bibitem{key1} Author Name. \textit{Title of the Work}. Journal/Conference, Year.
    \bibitem{key2} ...
\end{thebibliography}

\end{document}